\documentclass[12pt]{article}

\usepackage[margin=1in]{geometry}
\usepackage{setspace}
\onehalfspacing
\usepackage{graphicx}
\graphicspath{report_images/}
\usepackage{appendix}
\usepackage{listings}
\usepackage{float}
\usepackage{multirow}
\usepackage{amsthm}
% The next three lines make the table and figure numbers also include section number
\usepackage{chngcntr}
\counterwithin{table}{section}
\counterwithin{figure}{section}
% Needed to make titling page without a page number
\usepackage{titling}

% This section required for the code block stuff below
\usepackage{color}
\definecolor{codegreen}{rgb}{0,0.6,0}
\definecolor{codegray}{rgb}{0.5,0.5,0.5}
\definecolor{codepurple}{rgb}{0.58,0,0.82}
\definecolor{backcolour}{rgb}{0.95,0.95,0.92}

% Makes code blocks have shaded backgrounds and line numbers
\lstdefinestyle{mystyle}{
    backgroundcolor=\color{backcolour},   
    commentstyle=\color{codegreen},
    keywordstyle=\color{magenta},
    numberstyle=\tiny\color{codegray},
    stringstyle=\color{codepurple},
    basicstyle=\footnotesize,
    breakatwhitespace=false,         
    breaklines=true,                 
    captionpos=b,                    
    keepspaces=true,                 
    numbers=left,                    
    numbersep=5pt,                  
    showspaces=false,                
    showstringspaces=false,
    showtabs=false,                  
    tabsize=2
}
 
\lstset{style=mystyle}

% DOCUMENT INFORMATION =================================================
\font\titleFont=cmr12 at 11pt
\title {{\titleFont ECEN 421:  Embedded Systems Design \\ North Carolina Agricultural and Technical State University \\ Department of Electrical and Computer Engineering \\ Dr. C.A. Graves}} % Declare Title
\author{\titleFont  Chris Cannon \\ \titleFont Abbigail Waddell} % Declare authors
\date{\titleFont December 4, 2018}
% ======================================================================

\begin{document}

\begin{titlingpage}
\maketitle
\begin{center}
	Final Project Documentation
\end{center}
\end{titlingpage}

\tableofcontents

\pagebreak

\section{Introduction}
The purpose of this project is to create a design using a Texas Instruments MSP432 Launchpad and MKII Booster Pack, along with a Digilent Analog Discovery. This design should correctly identify diodes, capacitors, inductors, and resistors when these components are plugged in to our device's ports.

\section{Design Process}

\subsection{Research}
The first step in our research process was to ascertain the capabilities of the Analog Discovery. While the MSP432 was somewhat of a known quantity, neither one of us had significant experience with the Analog Discovery. Through this research, we found that the Analog Discovery had digital I/O pins and the ability to control them with JavaScript \cite{staticio} \cite{ioscript}.

\subsection{UML Activity Diagram}

\subsection{Block Diagrams}

\section{Conclusion}
% Add conclusion here

\pagebreak

\textbf{Appendices}

\begin{appendices}

\section{C Code for MSP432}

\lstinputlisting[language=C]{../main.c}

\section{JavaScript for Analog Discovery}

\lstinputlisting[language=java]{../script.js}

\end{appendices}

\begin{thebibliography}{9}

\bibitem{staticio}
  Using the Static I/O, \\
  Digilent Documentation, \\
  Digilent, Inc., \\
  \verb!https://reference.digilentinc.com/learn/instrumentation/! \\
  \verb!tutorials/ad2-static-io/start!
  
\bibitem{refman}
 Analog Discovery Technical Reference Manual, \\
 Digilent, Inc., \\
 Pullman, WA, \\
 2015 \\
 \verb![Online]! \\
 Available: \verb!https://reference.digilentinc.com/_media/! \\
 \verb!analog_discovery:analog_discovery_rm.pdf!
 
\bibitem{ioscript}
 Static IO Scripting Documents for Waveforms 3, \\
 Digilent Forums, \\
 Digilent, Inc. \\
 \verb!https://forum.digilentinc.com/topic/2150-staticio-scripting! \\
 \verb!-documentations-for-waveforms3/!
 
\bibitem{jsfunc}
 JavaScript Functions, \\
 w3schools, \\ 
 Refsnes Data, \\
 \verb!https://www.w3schools.com/js/js_functions.asp!

\end{thebibliography}
\end{document}